
\part{How to read this document}
\setcounter{section}{0}

% Front/summary page
\section{Summary (front) page}
Both the overall product's SQA adherence and per-task status codes are explained below: \\
\begin{tabular}{l L{10cm}}
    \greenbox{COMPLETE} & Task has been successfully completed and fulfills the project's SQA requirements, listed in \href{https://owncloud.indigo-datacloud.eu/index.php/s/yDklCrWjKnjutVA}{Deliverable D3.1} and \href{https://    project.indigo-datacloud.eu/projects/wp3/wiki/Extensions_to_SQA}{Extensions to Software Quality Assurance} documents. \\ 
    \redbox{NOT COMPLETE} & Task has not been completed, yet some missing required bits have not been provided. \\
    \graybox{IN PROGRESS} & Task has not been completed, but can proceed as it is. \\[0.1em]
    \graybox{WP3 PENDING} & Task has some pending work from WP3 side, meaning that the product team already submitted the required data but it has not been yet consumed by WP3. \\
\end{tabular}



% Task progress
\section{Task Progress}


\subsection{Code style}
\begin{tabular}{l L{10cm}}
    \graybox{Code style definition} & Name and link of the standard to which the product is adhered. \\
    \graybox{Community/de-facto standard} & Whether the adopted standard is community-wide accepted. \\
    \graybox{Exceptions} & Number of exceptions from the standard definition. \\
    \graybox{Richness} & Number of rules defined in the adopted standard. Additionally (whenever available) the \graybox{number of errors}, \graybox{number of warnings} documented in the standard will be displayed as well as the \graybox{link} to the latest definition. \\
\end{tabular}


\subsection{Unit testing}
This section will display the a) \graybox{trend graph} with the evolution of the code coverage over time and b) the \graybox{Cobertura report}, with the coverage results of different methods. Both are taken from the project's Jenkins continuous integration service. \\
\textit{Note}: resultant coverage value is the lowest of the ones for the different methods: packages, files, classes, lines, conditionals.


\subsection{Functional/Integration testing}


\subsection{GitBook documentation}
Whenever the documentation of the product is available at the project's GitBook repository, both the a) \graybox{link} to the documentation index and b) \graybox{type of documentation} provided will be displayed in the report.


\subsection{Configuration Management}
Whenever the product has an recipe to be deployed automatically the following information will be available: \\[0.5em]
\begin{tabular}{l L{10cm}}
    \graybox{Tool} & Configuration management tool used. \\
    \graybox{Manifest link} & URL pointing to the manifest/s. \\
    \graybox{Deployment level} & Whether \texttt{installation}, \texttt{configuration} or both. \\
    \graybox{Build status} & Current build status for the project's supported OS distributions. \\
\end{tabular}
